\documentclass{beamer}
\usepackage{amsmath, amssymb}
\usepackage{hyperref}
\usetheme{Madrid}

\title{LatinSquareGAP: A GAP Package for Working with Latin Squares}
\author{Seyyed Ali Mohammadiyeh}
\institute{GAP Days 2025}
\date{\today}

\begin{document}
	
	% Title Slide
	\begin{frame}
		\titlepage
	\end{frame}
	
	% Introduction
	\begin{frame}{What is a Latin Square?}
		\begin{block}{Definition}
			A \textbf{Latin square} of order $n$ is an $n \times n$ array filled with $n$ different symbols, each occurring exactly once in each row and each column.
		\end{block}
		\pause
		\begin{example}
			Example of a Latin square of order 3:
			\[
			\begin{bmatrix}
				1 & 2 & 3 \\
				3 & 1 & 2 \\
				2 & 3 & 1
			\end{bmatrix}
			\]
		\end{example}
	\end{frame}
	
	% Motivation
	\begin{frame}{Why Latin Squares?}
		\begin{itemize}
			\item Combinatorics and discrete mathematics
			\item Experimental design and statistics
			\item Algebra and group theory
			\item Applications in computer science and cryptography
		\end{itemize}
	\end{frame}
	
	% Package Overview
	\begin{frame}{About LatinSquareGAP Package}
		\begin{itemize}
			\item Developed for GAP system
			\item Tools to generate, count, and explore Latin squares
			\item Based on recursive backtracking algorithms
		\end{itemize}
	\end{frame}
	
	% Key Functions
	\begin{frame}{Key Functions}
		\begin{itemize}
			\item \texttt{LatinRow(n, r, c)}: Completes a row under column constraints
			\item \texttt{LatinCountRow(n, k, r, c)}: Counts completions of current row
			\item \texttt{LatinCount(n, k)}: Counts total Latin squares of order $n$
			\item \texttt{LatinList(n, c)}: Generates all Latin squares of order $n$
		\end{itemize}
	\end{frame}
	
	% Example Usage
	\begin{frame}{Example: Counting Latin Squares}
		\begin{itemize}
			\item To count Latin squares of order 3:
			\begin{verbatim}
				gap> LatinCount(3, []);
				12
			\end{verbatim}
			\item To generate and print all:
			\begin{verbatim}
				gap> L := LatinList(3, []);
				gap> Print(L, "\n");
			\end{verbatim}
		\end{itemize}
	\end{frame}
	
	% Random Square
	\begin{frame}{Selecting a Random Latin Square}
		\begin{itemize}
			\item No built-in random generator
			\item But you can do:
			\begin{verbatim}
				gap> L := LatinList(3, []);
				gap> RandomLatin := L[ Random(Length(L)) ];
			\end{verbatim}
		\end{itemize}
	\end{frame}
	
	% Algorithms
	\begin{frame}{How Does It Work?}
		\begin{itemize}
			\item Recursive backtracking
			\item Constraints: Unique symbols in rows and columns
			\item Efficient traversal of solution space
			\item Optimized for counting without generating all squares
		\end{itemize}
	\end{frame}
	
	% Installation
	\begin{frame}{Installation and Loading}
		\begin{block}{Installation}
			Clone the package into GAP's \texttt{pkg} directory.
		\end{block}
		\begin{block}{Loading}
			\begin{verbatim}
				gap> LoadPackage("LatinSquareGAP");
			\end{verbatim}
		\end{block}
	\end{frame}
	
	% License and Author
	\begin{frame}{About the Author}
		\begin{itemize}
			\item \textbf{Author:} Seyyed Ali Mohammadiyeh
			\item \textbf{License:} MIT
			\item \textbf{GitHub:} \url{https://github.com/BaseMax/LatinSquareGAP}
		\end{itemize}
	\end{frame}
	
	% Acknowledgements
	\begin{frame}{Acknowledgements}
		\begin{itemize}
			\item Thanks to GAP community
			\item Inspired by discussions during GAP Days
			\item Open for collaboration and contributions
		\end{itemize}
	\end{frame}
	
	% Thank You
	\begin{frame}
		\centering
		\Huge Thank You! \\
		\vspace{0.5cm}
		Questions?
	\end{frame}
	
\end{document}