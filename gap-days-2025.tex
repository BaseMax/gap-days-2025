\documentclass{report}
\usepackage{amsmath, amssymb, amsthm}
\usepackage{listings}
\usepackage[colorlinks=true,linkcolor=blue,urlcolor=blue,citecolor=blue]{hyperref}

\title{LatinSquare Manual}
\author{Seyyed Ali Mohammadiyeh}
\date{\today}

\begin{document}
	
	\maketitle
	\tableofcontents
	\clearpage
	
	\chapter{Introduction}
	This package provides functions to generate and count Latin squares using GAP. It offers the following capabilities:
	\begin{itemize}
		\item \textbf{Generation of Latin squares:} Construct all Latin squares of a given order.
		\item \textbf{Counting Latin squares:} Count the number of Latin squares without explicitly generating all of them.
		\item \textbf{Random Latin square selection:} Although no dedicated function is provided, you can generate all Latin squares and then select one at random.
	\end{itemize}
	
	\chapter{Functions}
	This package defines four global functions. Their implementations rely on recursive backtracking algorithms to construct Latin squares.
	
	\section{LatinRow}
	\textbf{Prototype:} \texttt{LatinRow(n, r, c)}\\[1mm]
	\textbf{Description:}  
	Computes all possible completions of a row for a Latin square of order \texttt{n}. The parameter \texttt{r} is a list of lists representing the current column restrictions, and \texttt{c} is the partially built row. When the length of \texttt{c} equals \texttt{n}, it returns the completed row as a valid candidate. 
	
	\textbf{Example:}
	\begin{lstlisting}
		gap> LatinRow(3, [[], [], []], []);
		[ [ 1, 2, 3 ], [ 1, 3, 2 ], ... ]
	\end{lstlisting}
	
	\section{LatinCountRow}
	\textbf{Prototype:} \texttt{LatinCountRow(n, k, r, c)}\\[1mm]
	\textbf{Description:}  
	Recursively counts the number of ways to complete the current row given:
	\begin{itemize}
		\item \texttt{n}: Order of the Latin square.
		\item \texttt{k}: The list of rows already fixed in the partial Latin square.
		\item \texttt{r}: The current state of column restrictions (i.e., the list of numbers already used in each column).
		\item \texttt{c}: The current row being constructed.
	\end{itemize}
	When a row is completed, the function calls \texttt{LatinCount} to continue the counting process.
	
	\section{LatinCount}
	\textbf{Prototype:} \texttt{LatinCount(n, k)}\\[1mm]
	\textbf{Description:}  
	Counts the total number of Latin squares of order \texttt{n} based on a partially constructed Latin square \texttt{k}. If \texttt{k} is empty, the function counts all possible Latin squares. This function is particularly useful when you want to know the number of Latin squares without generating each one explicitly.
	
	\textbf{Example:}
	\begin{lstlisting}
		gap> LatinCount(3, []);
		12
	\end{lstlisting}
	
	\section{LatinList}
	\textbf{Prototype:} \texttt{LatinList(n, c)}\\[1mm]
	\textbf{Description:}  
	Generates all Latin squares of order \texttt{n} by recursively constructing them row by row. The parameter \texttt{c} holds the rows of the Latin square constructed so far. When \texttt{c} is empty, the function generates the entire set of Latin squares of order \texttt{n}.
	
	\textbf{Example:}
	\begin{lstlisting}
		gap> L := LatinList(3, []);
		gap> Length(L);
		12
	\end{lstlisting}
	
	\chapter{Usage Examples}
	\section{Counting Latin Squares}
	To count the number of Latin squares of order 4, run:
	\begin{lstlisting}
		gap> LatinCount(4, []);
	\end{lstlisting}
	
	\section{Generating Latin Squares}
	To generate all Latin squares of order 3:
	\begin{lstlisting}
		gap> L := LatinList(3, []);
		gap> Print(L, "\n");
	\end{lstlisting}
	
	\section{Generating a Random Latin Square}
	Although the package does not include a dedicated random generator, you can select one at random from the list of all Latin squares:
	\begin{lstlisting}
		gap> L := LatinList(3, []);
		gap> RandomLatin := L[ Random(Length(L)) ];
		gap> Print(RandomLatin, "\n");
	\end{lstlisting}
	
	\chapter{Algorithm Overview}
	The package functions are based on recursive backtracking. The \texttt{Row} function constructs valid rows under the constraints of the Latin square (each symbol appears exactly once per row and column). The \texttt{CountRow} function leverages \texttt{Row} to accumulate counts without explicit list generation, while \texttt{LatinList} collects all completed Latin squares.
	
	\chapter{Installation and Loading}
	\textbf{Installation:}  
	Clone the repository into the \texttt{pkg} directory of your GAP installation. For example, if your GAP directory is located at \texttt{.../gap-4.X/}, place the LatinSquareGAP folder inside the \texttt{pkg} subdirectory.
	
	\textbf{Loading the Package:}  
	Start GAP and load the package with:
	\begin{lstlisting}
		gap> LoadPackage("LatinSquareGAP");
	\end{lstlisting}
	A confirmation message will indicate that the package has been loaded successfully.
	
	\chapter{License and Author}
	This package is released under the MIT License.\\[1mm]
	\textbf{Author:} Seyyed Ali Mohammadiyeh
	
	\chapter{Acknowledgements}
	Thank you to the GAP community for continuous support and to all contributors who helped in the development of this package.
	
\end{document}
