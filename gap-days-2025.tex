% pdflatex.exe -synctex=1 -interaction=nonstopmode -shell-escape "gap-days-2025".tex
% pdflatex -synctex=1 -interaction=nonstopmode -shell-escape "gap-days-2025".tex
\PassOptionsToPackage{colorlinks=true,linkcolor=blue,urlcolor=blue,citecolor=blue}{hyperref}
\documentclass{beamer}

\usepackage{amsmath, amssymb}
\usepackage{xcolor}
\usepackage{listings}

\definecolor{gapgreen}{HTML}{90A959}
\usetheme{default}
\usecolortheme[named=gapgreen]{structure}

\title{LatinSquare Manual}
\author[S. A. Mohammadiyeh]{Seyyed Ali Mohammadiyeh\\
Department of Pure Mathematics, Faculty of Mathematical Sciences\\
University of Kashan, Kashan 87317-53153, I. R. Iran\\
\texttt{alim@kashanu.ac.ir}, \texttt{max@std.kashanu.ac.ir}}
\date{\today}

\lstset{
    basicstyle=\ttfamily\small,
    keywordstyle=\color{gapgreen}\bfseries,
    commentstyle=\color{gray},
    stringstyle=\color{blue},
    showstringspaces=false,
    frame=single,
    breaklines=true
}

\setbeamertemplate{navigation symbols}{}

\setbeamertemplate{footline}{
  \leavevmode%
  \hbox{%
  \begin{beamercolorbox}[wd=.8\paperwidth,ht=2.5ex,dp=1ex,left]{author in head/foot}%
    \hspace{1em}\scriptsize Seyyed Ali Mohammadiyeh -- GAP Days 2025
  \end{beamercolorbox}%
  \begin{beamercolorbox}[wd=.2\paperwidth,ht=2.5ex,dp=1ex,right]{date in head/foot}%
    \scriptsize\insertframenumber{} / \inserttotalframenumber\hspace{1em}
  \end{beamercolorbox}}%
  \vskip0pt%
}

\begin{document}

\begin{frame}
  \titlepage
\end{frame}

\begin{frame}
  \frametitle{Table of Contents}
  \tableofcontents
\end{frame}

\section{What is a Latin Square?}

\begin{frame}
\frametitle{Definition}
A \textbf{Latin square} of order $n$ is an $n \times n$ array filled with $n$ different symbols, each occurring exactly once in each row and exactly once in each column.
\pause
\begin{itemize}
  \item Often, symbols are the integers $\{1, 2, \dots, n\}$.
  \item No symbol repeats in any row or column.
\end{itemize}
\end{frame}

\begin{frame}
\frametitle{Example of a Latin Square}
Here is a Latin square of order 3:
\[
\begin{bmatrix}
1 & 2 & 3 \\
2 & 3 & 1 \\
3 & 1 & 2 \\
\end{bmatrix}
\]
\pause
Each number appears once per row and once per column.
\end{frame}

\begin{frame}
\frametitle{More Examples (Order 4)}
\[
\begin{bmatrix}
1 & 2 & 3 & 4 \\
2 & 3 & 4 & 1 \\
3 & 4 & 1 & 2 \\
4 & 1 & 2 & 3 \\
\end{bmatrix}
\quad
\begin{bmatrix}
1 & 2 & 3 & 4 \\
3 & 4 & 1 & 2 \\
4 & 1 & 2 & 3 \\
2 & 3 & 4 & 1 \\
\end{bmatrix}
\]
\end{frame}

\begin{frame}
\frametitle{Applications}
Latin squares appear in:
\begin{itemize}
  \item \textbf{Statistics:} Experimental design.
  \item \textbf{Cryptography:} Encryption algorithms.
  \item \textbf{Puzzle design:} Sudoku is a partial Latin square!
  \item \textbf{Combinatorics:} Designs, finite geometry, etc.
\end{itemize}
\end{frame}

\begin{frame}
\frametitle{Historical Notes}
\begin{itemize}
  \item Studied by \textbf{Leonhard Euler} in the 18th century.
  \item Euler's conjecture on orthogonal Latin squares was famous (later disproven for $n=6$).
  \item Latin squares are related to group theory and quasigroups.
\end{itemize}
\end{frame}

\begin{frame}
\frametitle{Terminology}
\begin{itemize}
  \item \textbf{Order:} The size $n$ of the square.
  \item \textbf{Symbol set:} The $n$ elements used (commonly $1$ to $n$).
  \item \textbf{Orthogonal Latin squares (OLS):} Two Latin squares where each ordered pair occurs once.
\end{itemize}
\end{frame}

\begin{frame}
\frametitle{How Many Latin Squares?}
The number grows extremely fast:
\begin{itemize}
  \item $n = 1$: $1$
  \item $n = 2$: $2$
  \item $n = 3$: $12$
  \item $n = 4$: $576$
  \item $n = 5$: $161280$
  \item $n = 6$: over $8.9$ billion!
\end{itemize}
\pause
Exact formulas exist only for small $n$.
\end{frame}

\begin{frame}
\frametitle{Latin Squares in Sudoku}
\begin{itemize}
  \item A valid Sudoku solution is a Latin square with extra region constraints.
  \item Sudoku = Latin square + $3\times3$ box conditions.
\end{itemize}
\begin{center}
\includegraphics[width=0.5\textwidth]{img1}
\end{center}
\end{frame}

\section{Minimum Order for Latin Squares}
\begin{frame}
\frametitle{Minimum \( n \) for Latin Squares}
\begin{itemize}
  \item A Latin square is a grid filled with \( n \) rows and \( n \) columns, where each number appears exactly once in each row and column.
  \item \textbf{Order 1:} A 1x1 Latin square is trivially possible, as it just contains one element.\\
    Example: \(\begin{bmatrix} 1 \end{bmatrix}\)
  \item \textbf{Order 2:} A 2x2 Latin square is not possible. The problem is that there’s no way to place two numbers without repeating them in each row and column.
  \item \textbf{Order 3:} A 3x3 Latin square is the smallest non-trivial Latin square and is possible.
\end{itemize}
\end{frame}

\section{All Latin Squares of Order 1}
\begin{frame}
\frametitle{Latin Squares of Order 1}
The only Latin square of order 1 is:
\[
\begin{bmatrix}
1
\end{bmatrix}
\]
\end{frame}

\section{All Latin Squares of Order 2}
\begin{frame}
\frametitle{Latin Squares of Order 2}
It is impossible to create a Latin square of order 2, as no arrangement satisfies the row and column constraints.
\end{frame}

\section{All Latin Squares of Order 3}
\begin{frame}
\frametitle{Latin Squares of Order 3}
There are 12 Latin squares of order 3. They are:
\[
\begin{bmatrix}
1 & 2 & 3 \\
2 & 3 & 1 \\
3 & 1 & 2
\end{bmatrix},
\quad
\begin{bmatrix}
1 & 3 & 2 \\
2 & 1 & 3 \\
3 & 2 & 1
\end{bmatrix},
\]
and 10 more.\\
(You can display these squares across multiple slides if necessary.)
\end{frame}

\section{All Latin Squares of Order 4}
\begin{frame}
\frametitle{Latin Squares of Order 4}
There are 576 Latin squares of order 4. Due to their large number, only a few are shown here:
\[
\begin{bmatrix}
1 & 2 & 3 & 4 \\
2 & 3 & 4 & 1 \\
3 & 4 & 1 & 2 \\
4 & 1 & 2 & 3
\end{bmatrix},
\quad
\begin{bmatrix}
1 & 2 & 4 & 3 \\
2 & 3 & 1 & 4 \\
3 & 4 & 2 & 1 \\
4 & 1 & 3 & 2
\end{bmatrix},
\]
and 574 more.\\
(You can display these squares across multiple slides if necessary.)
\end{frame}

\section{All Latin Squares of Order 5}
\begin{frame}
\frametitle{Latin Squares of Order 5}
There are 161280 Latin squares of order 5. Here are some examples:
\[
\begin{bmatrix}
1 & 2 & 3 & 4 & 5 \\
2 & 3 & 4 & 5 & 1 \\
3 & 4 & 5 & 1 & 2 \\
4 & 5 & 1 & 2 & 3 \\
5 & 1 & 2 & 3 & 4
\end{bmatrix},
\quad
\begin{bmatrix}
1 & 2 & 3 & 5 & 4 \\
2 & 3 & 4 & 1 & 5 \\
3 & 4 & 5 & 2 & 1 \\
4 & 5 & 1 & 3 & 2 \\
5 & 1 & 2 & 4 & 3
\end{bmatrix},
\]
and 161278 more.\\
(You can display these squares across multiple slides if necessary.)
\end{frame}

\begin{frame}
\frametitle{Construction Methods}
Several techniques:
\begin{itemize}
  \item \textbf{Cyclic method:} Rotate rows.
  \item \textbf{Backtracking:} Recursive filling (used in our package).
  \item \textbf{Group-based construction:} Use permutation groups.
\end{itemize}
\end{frame}

\begin{frame}
\frametitle{Recap}
\begin{itemize}
  \item Latin squares are $n \times n$ grids with unique symbols in each row and column.
  \item They have many real-world applications.
  \item Enumeration and generation are key challenges.
\end{itemize}
Next, we'll look at how the GAP package works with Latin squares.
\end{frame}

\section{Introduction}
\begin{frame}
\frametitle{Introduction}
This package provides functions to generate and count Latin squares using GAP. It offers:
\begin{itemize}
  \item \textbf{Generation of Latin squares:} Construct all Latin squares of a given order.
  \item \textbf{Counting Latin squares:} Count them without generating all explicitly.
  \item \textbf{Random selection:} Generate all and select one at random.
\end{itemize}
\end{frame}

\section{Functions}
\begin{frame}[fragile]
\frametitle{LatinRow}
\textbf{Prototype:} \texttt{LatinRow(n, r, c)}\\
\textbf{Description:} Computes valid completions of a row for a Latin square of order \texttt{n}.
\begin{lstlisting}
gap> LatinRow(3, [[], [], []], []);
[ [ 1, 2, 3 ], [ 1, 3, 2 ], ... ]
\end{lstlisting}
\end{frame}

\begin{frame}[fragile]
\frametitle{LatinCountRow}
\textbf{Prototype:} \texttt{LatinCountRow(n, k, r, c)}\\
\textbf{Description:} Recursively counts valid completions of the current row.
\begin{itemize}
  \item \texttt{n}: Order of the square.
  \item \texttt{k}: Rows fixed so far.
  \item \texttt{r}: Column restrictions.
  \item \texttt{c}: Current row.
\end{itemize}
\end{frame}

\begin{frame}[fragile]
\frametitle{LatinCount}
\textbf{Prototype:} \texttt{LatinCount(n, k)}\\
\textbf{Description:} Counts all Latin squares of order \texttt{n}, optionally using partial data.
\begin{lstlisting}
gap> LatinCount(3, []);
12
\end{lstlisting}
\end{frame}

\begin{frame}[fragile]
\frametitle{LatinList}
\textbf{Prototype:} \texttt{LatinList(n, c)}\\
\textbf{Description:} Generates all Latin squares of order \texttt{n}.
\begin{lstlisting}
gap> L := LatinList(3, []);
gap> Length(L);
12
\end{lstlisting}
\end{frame}

\section{Usage Examples}
\begin{frame}[fragile]
\frametitle{Counting Latin Squares}
To count the number of Latin squares of order 4:
\begin{lstlisting}
gap> LatinCount(4, []);
\end{lstlisting}
\end{frame}

\begin{frame}[fragile]
\frametitle{Generating Latin Squares}
To generate all Latin squares of order 3:
\begin{lstlisting}
gap> L := LatinList(3, []);
gap> Print(L, "\n");
\end{lstlisting}
\end{frame}

\begin{frame}[fragile]
\frametitle{Random Latin Square}
To randomly select one:
\begin{lstlisting}
gap> L := LatinList(3, []);
gap> RandomLatin := L[ Random(Length(L)) ];
gap> Print(RandomLatin, "\n");
\end{lstlisting}
\end{frame}

\section{Algorithm Overview}
\begin{frame}
\frametitle{Algorithm Overview}

Backtracking, a problem-solving technique in computer science, is generally considered a good and powerful approach for exploring multiple possibilities systematically and finding solutions, especially when dealing with constraints and complex problems. 

This package uses recursive backtracking:
\begin{itemize}
  \item \texttt{LatinRow}: Valid row generation.
  \item \texttt{LatinCountRow}: Counts completions via recursion.
  \item \texttt{LatinList}: Builds full Latin squares.
\end{itemize}
\end{frame}

\section{Installation}
\begin{frame}[fragile]
\frametitle{Installation and Loading}
\textbf{Installation:}  
Place the package inside GAP's \texttt{pkg} directory.\\
\textbf{Loading:}
\begin{lstlisting}
gap> LoadPackage("LatinSquareGAP");
\end{lstlisting}
You should see a confirmation message.
\end{frame}

%\section{License and Author}
%\begin{frame}
%\frametitle{License and Author}
%\textbf{License:} MIT License\\[1mm]
%\textbf{Author:} Seyyed Ali Mohammadiyeh
%\end{frame}

\section{Q\&A}
\begin{frame}
	\frametitle{Thank You!}
	\centering
	\Large Thank you so much for your attention!\\[1em]
	\Large Any questions?
\end{frame}

\section{Acknowledgements}
\begin{frame}
\frametitle{Acknowledgements}
Thanks to the GAP community and contributors who supported this work!
\end{frame}

\end{document}
